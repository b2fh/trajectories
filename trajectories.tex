\documentclass[11pt]{article}
\pdfoutput=1
\usepackage{jcappub}
\usepackage{aas_macros}
\usepackage{graphics,graphicx}
\usepackage{amsmath,amssymb}
\graphicspath{{figures/}}

\begin{document}

\title{Primordial Potential Reconstruction Constrained by a Measured Tensor Amplitude and the Preheating Regime}
\author[a]{J. Richard Bond}
\author[a,b]{Jonathan Braden}
\author[c]{Andrei Frolov}
\author[a]{Zhiqi Huang}
\author[d]{Pascal Vaudrevange}
\affiliation[a]{CITA}
\affiliation[b]{Dept. of Physics, U of T}
\affiliation[c]{Simon Fraser University}
\affiliation[d]{Germany}

\emailAdd{bond@cita.utoronto.ca}
\emailAdd{jbraden@cita.utoronto.ca}
\emailAdd{frolov@sfu.ca}
\emailAdd{zqhuang@cita.utoronto.ca}

\abstract{The shapes of the primordial scalar power spectra are the key quantities to unravel the physics of the inflationary epoch. Blahblah...}

\date{\today}
\maketitle

\section{Introduction}

The cosmic microwave background radiation is a unique window into the physics of energy scales above TeV. Progress towards harvesting its information content has been steady on the experimental side, starting with the COBE experiment \cite{COBE1996}, continuing with balloon borne experiments such as Boomerang \cite{Boomerang2001, Boomerang2003}, ground-based experiments such as ACT \cite{ACT2013, ACT2014}, SPT \cite{SPT2013, SPT2014} and BICEP \cite{BICEP2}, and the satellites WMAP \cite{WMAP9Maps, WMAP9Cosmology} and Planck \cite{Planck2013Overview, Planck2013PowerSpectra, Planck2013Parameters}. They delivered a picture of a universe that is extremely homogeneous: Gaussian fluctuations sit on top of a uniform background of photons that stream to us from the surface of last scattering at redshift about $1100$, with an amplitude of about $10^{-5}$. Decomposing this image of the microwave sky into spherical harmonics shows an angular power spectrum whose features are well understood. Acoustic oscillations in the primordial photon-baryon fluid freeze out at the surface of last scattering, with the photons streaming (almost) freely towards us, showing the familiar acoustic peaks in the angular power spectrum. Their locations and relative amplitudes allow us to infer the energy content of the universe.

While experimental advances have been formidable, the exact properties of microscopic theories responsible for the inflationary period remain elusive. A plethora of different scalar field potentials have been suggested, and many of them are compatible with current observations. From a theoretical perspective, the best that can be said is that there are many scalar fields in string theory. As for their potentials, not much is known. Even though some advances have been made in building models of inflation based on or inspired by string theory \cite{KKLT, Blanco-Pillado:2004ns, Blanco-Pillado:2006he, KKLMMT, CQ, GKP, Silverstein:2008sg}, the vast majority of the string landscape is still unchartered.

The usual approach in inflationary model building is to specify parameterized scalar field potentials motivated by fundamental particle physics, proceed to derive the dynamical histories and compare with cosmic observations model by model using only a few parameters such as the scalar amplitude and spectral index, a ``top--down'' framework. With the increasing emphasis on complex potential landscapes dotted by many local minima with varied structures surrounding them, large ensembles of possible trajectories could arise. Viable trajectories are determined probabilistically, from theoretical ``prior'' information.

Alternatively, one can take a bottom-up approach from the data, whether with very broad error bars, as in anthropic considerations, or of the increasingly high precision sort offered by the Cosmic Microwave Background (CMB) and Large Scale Structure (LSS) observations. Although there is much art in deciding what theoretical priors to impose, based on concepts such as ``baroqueness'' of the effective potentials allowed, it is best to allow the priors to be broadly defined and let the data decide what is allowed and what not. It is a well-trod path in concept and one we develop further in this paper.

Traditionally the inflationary dynamics are described in the slow roll approximation, assuming that the inflaton field rolls slowly (or with small acceleration) down its potential. The observables -- the primordial tensor and scalar power spectra -- can be computed as a function of the slow roll parameters, typically compressed into a restricted set of spectral parameters, such as the scalar amplitude, the tensor-scalar ratio, the scalar power law index, its running and the tensor index. The values of the slow roll parameters, i.e. the parameters of the scalar potential, can thus be inferred using Markov Chain Monte Carlo (MCMC) methods, comparing the model predictions with observations. 

In this paper, we advocate a different approach. Instead of starting off with (a family) of scalar field potentials, we aim at reconstructing the shape of the primordial tensor and scalar power spectra independent of theoretical priors (in the sense of expecting certain shapes of the scalar field potential). Instead, we let observational data almost freely decide the shapes of the power spectra, using two different paradigms. The goal is to reconstruct -- in a model-independentl way -- nontrivial features in the primordial power spectra, if there are any, without imposing strong theoretical priors.

To this end, we consider spectra features produced by some non-standard processes. Examples of such models are particle production during inflation \cite{Barnaby2009a, Barnaby2009b}, cosmological fluctuations from preheating \cite{Bond2009}, and ``curvaton'' models \cite{Linde1996,Lyth2003}. In these models the consistency relation between tensor and scalar spectra breaks down. We hence let the scalar and tensor spectrum vary independently. In order to generate spectra with a finite number of parameters, we have to impose certain smoothing conditions (priors), implicitly defined by the interpolation method. We vary the interpolation method to show that the dependence on the smoothness prior is weak, provided that proper number of knots are used\footnotemark. 

Secondly, we consider spectra produced by single-field inflation with exotic features in the potential that can break down the slow roll approximation. Some extreme examples are discussed in e.g. Ref.~\cite{Starobinsky1998}. For these models we will impose the single field consistency condition, in effect forcing all observables to be derivable from a single real scalar field potential. Once again, we will strive to be agnostic about the shape of the potential. All that we require is an inflationary period. 

\footnotetext{The criterion is that the reduced $\chi^2$ is not significantly smaller than $1$, or expressed in the Bayesian language, the Bayesian evidence is not too low.}

This paper is structured as follows. Blahblah...

\section{Reconstructing Primordial Trajectories if the Tensor-to-Scalar  is Measured}

\subsection{Scalar Curvature Power Spectrum }

The problem with letting $r$ float. 

\subsection{Acceleration Histories}

\subsection{Inflaton Potential Reconstructions}

\section{Connecting Constrained Potentials in the Observable Range to the Potential in the Heating Regime }

B-modes help to really nail down the form of the potential during the
inflationary phase.  
Unfortunately, without a precise model to consider, the inflationary
dynamics are effectively independent of the reheating dynamics.
In particular, the inflaton condensate can break up via a variety of
different mechanisms depending on the ``shape'' of the potential
(tachyonic/spinodal instability, perturbative decay, amplification of
additional fields via parametric resonance followed by rescattering,...). 
The question is then how to relate the inflationary potential to the
potential at the minimum that is relevant for (p)reheating.
This seems rather difficult in the general case, since the ``shape'' of the
potential could change rather dramatically between inflation and
post-inflationary oscillations.  As some examples, in the single field case
the potential could be flattened (or steepened) at large field values, or
there could be a waterfall transition to end inflation.  This allows for a
large hierarchy between the effective mass of the inflaton during inflation
and the effective mass of the field as it oscillates.  Therefore, in the
completely general case it seems unlikely that the inflationary dynamics
could tell us much about the preheating dynamics.

Roughly, we want to have $P(\{a_i\},\{b_i\}|r)$ (or perhaps
$P(\{a_i\}|\{b_i\},r)$ for the case that the inflationary part is well
constrained)
where the $a_i$'s are some parameters that characterize the minimum of the
potential, and the $b_i$'s characterize the inflationary part of the
potential.
One possible approach is simply to expand the potential in a Taylor series
around the minimum (including several fields not just the inflaton).
It might also be interesting to expand in some other set of basis functions
(Chebychev's, Rational Chebyshev's, etc.).
This would at least give a definite way to define the coefficients in the
expansion.

One very unsophisticated take on this would be that a large value of $r$
means we have some sort of large-field model of inflation (even $m^2\phi^2$
for example).
If something like a simple polynomial is a good fit to the data during
inflation, then it is not unreasonable to consider the implications of
simply using a Taylor expansion around the minimum of the potential for the
preheating as well as the inflationary dynamics.
Naively, I would say this puts us into preheating regime that is described
by the billiard picture (although this might not be true).
Since we have a large field model, the effective couplings of the inflaton
to other fields will be enhanced by $(\phi_{end}/m_{eff})^{power} \sim
(M_P/m_{eff})^{power}$ and the couplings in the model will have to be very
small to be in the perturbative regime.
If this is true, then preheating and the eventual development of
nonlinearly interacting inhomogeneous fields is the correct description of
the end of inflation (modelling with some simple decay models isn't), which
is of course exactly what we've been working on.

The cleanest way to connect the B-mode results to preheating would be to
just assume that the Higgs is the inflaton (assuming that some appropriate
non-minimal terms in the Lagrangian or quantum effects allow it to match
the data).
Then we at least know all of the interactions as the Higgs is oscillating
around its minimum (and there are couplings to the gauge bosons of the form
$g^2h^2Z^2$ so it is not that dissimilar from $\lambda\phi^4 +
g^2\phi^2\chi^2$ billiards.
There are a couple papers on Higgs preheating (0812.4624 and 0812.3622),
but no lattice/nonlinear analysis that I know of.
Of course, having isocurvature modes in the gauge fields is probably
severely constrained so this might not be that interesting, and dealing
with the fermions is nontrivial.


\section{Flattening Preheating Potentials via Conformal Transformations}

\begin{equation} V(\phi, \chi) = 1/4 \lambda \phi^4  - 1/2 \xi \phi^2 R  1/2 g^2 \phi^2 \chi^2 \, .
\end{equation}

Conformal V-flattening as in SBB89, explored there as Higgs inflation with variable Planck mass models. 

The kinetic piece transforms as 

The field going into canonical form is 

The potential is transformed  as 

SBB considered large $\xi$. 

This story was recently picked up by Kallosh and Linde, who considered $\xi = 1/6 -\Delta$, where $\Delta$ is small. 

This should be compared with the heavy field V-flattening described in Dong, Horn, Silverstein, Westphal 2011, with $\phi^{2n}, \ n<1$ 
via heavy field trough driving light inflaton $V_{eff}$ yielding 
$r = 8n/(N_I +n/3) 1-ns  = (n+1)/(N_I-n/6)$, as in bh95. These forms are P13 OK

An example is one of the first  monodromy cases considered, SW08, with $p=1/3$. Later the form adopted in  MSW08 had $p=1/2$ and a $ \cos $ term associated with a shift symmetry. Another example is  roulette inflation (Kahler moduli) BKKV, where V-flat appears naturally. 
 
\begin{figure}
\includegraphics[width=0.5\linewidth]{{{potseq-1}}}
\caption{Conformally transformed potentials $U(\psi )$ for a sequence of $\xi$, with $\lambda$ fixed. }
\end{figure}

\begin{figure}
\includegraphics[width=0.5\linewidth]{{{potseq-2}}}
\caption{Conformally transformed potentials for a sequence of $\xi$, with $\lambda/\xi^2$ fixed. }
\end{figure}

\begin{figure}
\includegraphics[width=0.5\linewidth]{{{af_phase_portrait_ximinus1_shading_N_phi_pi}}}
\caption{Phase portrait for the class of conformally flattened potentials considered here, with $\xi = -1$. The shading denotes $\ln a (\phi ,\pi_\phi )$. }
\end{figure}

\section{Caustics from Ballistic Trajectories}

Development of caustics in ballistic trajectories for
\begin{equation}
  \mathcal{L} = \frac{R}{2} - \frac{1}{2}\partial_\mu\phi\partial^\mu\phi - \frac{1}{2}\partial_{\mu}\chi\partial^{\mu}\chi + \xi\phi^2R - \frac{g^2}{2}\phi^2\chi^2
\end{equation}
{\bf Double check normalizations and coupling constant definitions}

\begin{figure}
\includegraphics[width=0.5\linewidth]{{{mom_zeta0.5}}}
\caption{Development of caustics for $\xi=0.5$}
\end{figure}

\begin{figure}
\includegraphics[width=0.5\linewidth]{{{mom_zeta1}}}
\caption{Development of caustics for $\xi=1$}
\end{figure}

\begin{figure}
\includegraphics[width=0.5\linewidth]{{{mom_zeta2}}}
\caption{Development of caustics for $\xi=2$}
\end{figure}

\section{Full Lattice Evolution}
\begin{figure}
  \includegraphics[width=0.5\linewidth]{{{energy_partition}}}
  \caption{Distribution of energy between various components for the starobinski model coupled to a second scalar field.  I need to check back in the code to figure out exactly which model and parameters were used here.}
\end{figure}


\bibliographystyle{JHEP}  
\bibliography{trajectories}



\end{document}
